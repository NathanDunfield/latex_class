\documentclass[grant, tikz, noaddress]{nmd-article}
\trimpages{1cm}
\setlength{\parindent}{0pt}
\usepackage[scaled=0.82]{beramono}
% Copied from the thmtools documentation source code
\usepackage{listings}
\lstloadlanguages{[LaTeX]TeX}
\lstset{language=[LaTeX]TeX, 
        basicstyle=\footnotesize\ttfamily,
        keywordstyle=\mdseries, aboveskip=2\parsep,
        framesep=2ex, 
        xleftmargin=1em,
        framexleftmargin=1em,
        backgroundcolor=\color{nmdlight!30},
        rulecolor=\color{white},
        frame=lines}

\lstMakeShortInline[basicstyle=\normalsize\ttfamily]{|}

\newcommand{\exampleviacols}[1]{%
  \begin{minipage}{0.60\textwidth}
    \lstinputlisting{%
      tikz_tips_snippets/#1.tex
  } 
  \end{minipage} \quad
  \begin{minipage}{0.37\textwidth}
    \input{tikz_tips_snippets/#1.tex}
  \end{minipage}
}


\newcommand{\exampleviarows}[1]{%
  \begin{minipage}{\textwidth}
    \lstinputlisting{%
      tikz_tips_snippets/#1.tex
    }

    \vspace{0.5ex}

    \input{tikz_tips_snippets/#1.tex}

    \vspace{1.5ex}
  \end{minipage}
}



\title{TikZ tips and tricks}
\author{Nathan M. Dunfield}

\begin{document}
\maketitle


This document assumes that |nmd/graphics| has been loaded with the
|tikz| option enabled.  The |nmdstd| style includes my preferred line
styles and arrows.

\exampleviacols{arrows}

Note that |nmd/graphics| also loads the following TikZ libraries:

\begin{lstlisting}
\usetikzlibrary{calc}
\usetikzlibrary{positioning}
\usetikzlibrary{intersections}
\usetikzlibrary{arrows.meta}
\usetikzlibrary{decorations.markings}
\usetikzlibrary{decorations.pathreplacing}
\end{lstlisting}


\section{Recommended workflow}

\begin{enumerate}
  \item On paper, draw sketches and the finalize overall design as much as
    possible. The more that is fixed now, the easier this will be.
    
  \item Scan the final drawing and crop in e.g.~Preview.  Go to the
    master document and do something like the following to determine
    the rough size of the figure.

\begin{lstlisting}
\begin{figure}
  \begin{center}
    \includegraphics[width=4.5cm]{scan}
  \end{center}
  \caption{A caption}
  \label{fig:test}
\end{figure}    
\end{lstlisting}

  \item Create a file |plots/afig.tex| where the TikZ
    code will live containing

\begin{lstlisting}
\begin{tikzpicture}[nmdstd]
  \node[above right, opacity=0.3] 
        at (0, 0) {\includegraphics[width=4.5cm]{scan}};  
  \measurementgrid{5}{5};
\end{tikzpicture}
\end{lstlisting}

    and setup a minimal document to use whilst
    programming the figure

\begin{lstlisting}
\documentclass[margin=0.5cm]{standalone}
\usepackage{nmd/math}
\usepackage[tikz]{nmd/graphics}
\begin{document}
  \input plots/afig.tex
\end{document}
\end{lstlisting}

  \item Now hack away at |afig.tex| while compiling the minimal
    document.  A good approach is first to define |\coordinates| for
    key points using the grid and plot them

\begin{lstlisting}
  \coordinate (A) at (0.8, 0.4);
  \coordinate (B) at (4.5, 2.5);
  \coordinate (C) at (0.6, 4.6);

  \foreach \p in {A,B,C}{
    \node[opacity=0.2] at (\p) {\p};
  }
\end{lstlisting}
  
    Now use convex combinations and barycentric coordinates to define
    even more points

\begin{lstlisting}
    \coordinate (D) at ($(A)!0.25!(B)$);
    \coordinate (center) at (barycentric cs:A=0.5,B=0.5,C=0.5);
\end{lstlisting}

\end{enumerate}

\section{Drawing B\'ezier curves efficiently}

Visualize the control points via the |show controls| style and specify
them relative to the endpoints using polar coordinates as in the
second example.

\exampleviarows{controls}

There is a whole package |spath| for working with paths as
objects, but overall it seems like the best approach if you want to
reuse a path is just to declare a macro.

\exampleviarows{paths}

Even better than in the previous examples, you can use my
|easybezier| macro to specify the control points using polar
coordinates oriented with respect to the line between the end points
so that the distance between the endpoints is scaled to be one unit.

\exampleviacols{easy}

\section{Doing things along paths}

Often we want to do something along a path, like add an arrow or a
coordinate or add a tangent or normal vector.  This is done with the
decorations/markings functionality of TikZ, which is a bit verbose, so
I made shortcuts for some of these.

\exampleviarows{oncurve}



\end{document}
